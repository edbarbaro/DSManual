\documentclass[10pt]{PhDthesisPSnPDF}%
\usepackage{thesis_body}
\usepackage{pdfpages}
%%%%%%%%%%%%%%%%%%%%%%%%%

\usepackage{amsmath}%
\usepackage{amstext}%
\usepackage{amsfonts}%
\usepackage{amssymb}%
\usepackage{graphicx}
%\usepackage[latin1]{inputenc} 
\usepackage[T1]{fontenc}
\usepackage{authblk}
%\usepackage[bottom]{footmisc}
%\usepackage[top=3cm,left=2cm,right=2cm,bottom=2.5cm]{geometry} 
\usepackage[parfill]{parskip}
%\usepackage{titlesec}

%------------------------------------------------------------------
\hfuzz5pt % Don't bother to report overfull boxes < 5pt
\newtheorem{theorem}{Theorem} 
\newtheorem{corollary}[theorem]{Corollary}
\newtheorem{definition}[theorem]{Definition}
\newtheorem{example}[theorem]{Example}
\newtheorem{exercise}[theorem]{Exercise}
\newtheorem{lemma}[theorem]{Lemma}
\newtheorem{proposition}[theorem]{Proposition}
\pagestyle{plain}
%%% --------------------------------------------------------------
\DeclareGraphicsExtensions{.jpg,.pdf,.mps,.png}

%\setcounter{tocdepth}{6}
%\setcounter{secnumdepth}{6}
%\setcounter{chapter}{1}

\title{\Huge{An introduction to \\ applied data science} \\ 
\rule{100mm}{5pt} \\
\Large{A 5-day workshop towards practical \\ big-data applications and analytics}}
%  with practical experience extracting value from big data
\author{\Large{Eduardo Barbaro}}
\author{\Large{Coen Jonker}}
\affil{\normalsize Data Scientistists at Mobiquity Inc}
\date{\today}

% ----------------------------------------------------------------------
% turn of those nasty overfull and underfull hboxes
\hbadness=10000
\hfuzz=50pt


%%%%%%%%%%%%%%%%%%%%%%%%%%%%%%%%%%%%%%%%%%%%%%%%%
% Start-up
%%%%%%%%%%%%%%%%%%%%%%%%%%%%%%%%%%%%%%%%%%%%%%%%%
\begin{document}

\renewcommand\baselinestretch{1}
\baselineskip=14pt plus1pt

\pagenumbering{roman}    %  roman numbering for all preamble pages
\maketitle

% ----------------------------------------------------------------

\noindent{\textbf{About the workshop and this manual:}} \\

Welcome to ``An introduction to applied data science'' a 5-day original workshop thought and written by Eduardo Barbaro and Coen Jonker. At the end of this week you will be able to (i) understand the basics of data science, (ii) mathematically describe data, and (iii) munge\footnote{describes the constructive operation of tying together systems and interfaces that were not specifically designed to interoperate} it into an easy and communicable form. As the course develops, we will teach you how to access and look at data in innovative ways, as well as to extract value from big data. We aim a significant part of this training to help you create and interpret a diverse set of numerical models, as well as to calculate descriptive analytics while fully understanding their meaning. We also focus on presentation, and how to show your results in beautiful smart charts and tables.

Although all the computations will be performed in a cloud environment, we will guide you through all the necessary steps to install any software necessary for this course in your local machines. We take advantage of a cloud computing environment to get to run our code in a fast, secure, and controlled domain.

This manual is divided in 11 independent short chapters. In Chapter \ref{DS} we set the stage with a short introduction to then answer together a bold question: ``what is data science?''. In Chapter \ref{basics} we will spend some time to recap some basic mathematical concepts needed throughout the entire workshop. Chapters \ref{P1} and \ref{P2} are devoted to (important) practicalities, such as installing software/libraries and getting familiar with the cloud environment. We also teach you some basic programming skills, so we are all on the same page. We move forward to our first data harvest, in Chapter \ref{harvesting}. We teach you how to manipulate data, as well as to properly describe and clean it. Chapter \ref{numerics} covers the basics of designing a smart numerical experiment. 

Chapters \ref{MLearn} to \ref{bigData} cover more advanced topics such as machine learning tools, databases (SQL and no-SQL) as well as a mapreduce implementation (Hadoop). In particular, in Chapter \ref{bigData} we show in detail how to load, transform, and extract value from real big data. Finally, in Chapter \ref{Commun} we come back to basics to show you rudiments of plotting and how to communicate your results. 

We wish you a pleasant learning!

\newpage
\thispagestyle{empty}
\mbox{}
\cleardoublepage

% Counting	
%\setcounter{page}{1}
\setcounter{tocdepth}{2}


%%%%%%%%%%%%%%%%%%%%%%%%%%%%%%%%%%%%%%%%%%%%%%%%%
% Table of contents
%%%%%%%%%%%%%%%%%%%%%%%%%%%%%%%%%%%%%%%%%%%%%%%%%

%{\baselineskip=.8\baselineskip % shortens the spacing between lines in the TOC

\setcounter{secnumdepth}{3} % organisational level that receives a numbers
\setcounter{tocdepth}{2}    % print table of contents for level 3
\tableofcontents            % print the table of contents


%%%%%%%%%%%%%%%%%%%%%%%%%%%%%%%%%%%%%%%%%%%%%%%%%
% List of Figures and Tables
%%%%%%%%%%%%%%%%%%%%%%%%%%%%%%%%%%%%%%%%%%%%%%%%%
	
%\addcontentsline{toc}{chapter}{List of Figures}
%\listoffigures 
%\cleardoublepage
%\addcontentsline{toc}{chapter}{List of Tables}
%\listoftables 
%\cleardoublepage

%%%%%%%%%%%%%%%%%%%%%%%%%%%%%%%%%%%%%%%%%%%%%%%%%
% Chapters 
%%%%%%%%%%%%%%%%%%%%%%%%%%%%%%%%%%%%%%%%%%%%%%%%%
\pagenumbering{arabic}	 %  arabic numbering for all preamble pages

% Introduction
\newpage
\chapter{A brief history of data science}\markboth{A brief history of data science}{A brief history of data science}\label{DS}
In this Chapter we briefly try to shed some light on the question ``what is data science?''. First of all, it is important to realize that data science is not a new concept. There are many complex definitions out there, but in our view, data science is simply the coupling of very well-established disciplines, such as mathematics and statistics, with a relatively young discipline, computer sciences. 

Have a look at Fig. \ref{fig:Analytics}. It shows the evolution of 5 fundamental disciplines related to data science (Mathematics, Statistics, visualization, technology, computer science). Note that this cartoon depicts events as old as the ``invention'' of modern calculus by Newton/Leibniz (in the 17$^{th}$ century) or foundation of probability theory by Cardano (in the 16$^{th}$ century). You can see that back in those times all the disciplines were really self-contained areas without much interconnection. As we enter the 20$^{th}$ century things start to get more interconnected. Statistics and Mathematics cannot be separated so clearly any more (e.g. stochastic models, survival models), and computer sciences and technology started to gain more terrain (e.g. relational databases, tree-based models). 
\newpage
\begin{figure}[h]
	\begin{center}
			\includegraphics[scale=0.25]{Parts/history/HistoryCH1.png}
	\end{center}
	\caption{Most important events in the history of data science. \textit{Credit: Mamatha Upadhyaya}}
	\label{fig:Analytics}
\end{figure} 

As early as 1962, John Wilder Tukey, an American Mathematician, wrote in his ``The future of data'':
\begin{quotation}
\textit{For a long time I thought I was a statistician, interested in inferences from the particular to the general. But as I have watched mathematical statistics evolve, I have had cause to wonder and doubt. I have come to feel that my central interest is in data analysis. Data analysis, and the parts of statistics which adhere to it, must take on the characteristics of science rather than those of mathematics data analysis is intrinsically an empirical science.}
\end{quotation} 
 
That served as inspiration to Tukey who, in 1977, published his most well-known book ``Exploratory Data Analysis''. In that book, he argued that exploratory and confirmatory data analyses must proceed alongside. Also in the 1970's, Peter Naur, Danish computer science pioneer, published the important ``Concise Survey of Computer Methods''. In this book, the term data science was used for the first time. According to him, data science can be defined as:
\begin{quotation}
\textit{The science of dealing with data, once they have been established, while the relation of the data to what they represent is delegated to other fields and sciences.}
\end{quotation} 

In the beginning of the 21$^{st}$ century all the disciplines related to data science merged. In 2008, Jeff Hammerbacher (Facebook) and DJ Patil (LinkedIn) used the terminology ``Data scientist'' to define their work and teams. Since then, the use of this terminology has fully infiltrated the vernacular, and did not stop growing. 

Today, data science and computer sciences (through machine learning) have been put together as almost synonyms. There are new terminologies appearing every year, such as Deep Learning, Big Data, Data Mining. Today, we understand that data science and big data do not mean (just) ``lots of data''. Instead, it means the creation of a new paradigm in how we do analysis and combine the use of our traditional tools (Mathematics and Statistics) with the technology available nowadays.
\cleardoublepage

\newpage
\chapter{Some basics and beyond}\markboth{Some basics and beyond}{Some basics and beyond}
\label{basics}
Before we start with some hands-on data science, we think it is fundamental to ensure we are all on the same page on the Mathematics and Statistics we need throughout this course. But don't worry, it is nothing fancy. We, by no means, have the intent to \textit{formally} teach you any Mathematics. Instead, we just want to refresh some of the concepts we already know, but may have forgotten throughout the years.

\section{Mean, median, and standard deviation}

We would like to start with a good look at Fig. \ref{fig:boxplot}. This type of graphical representation is called a box-plot\footnote{The box-plot was created by John Wilder Tukey, the American Mathematician we talked about in Chapter 1.}. It shows the variation in miles per gallon (MPG) for cars with different numbers of cylinders\footnote{This data was extracted from the 1974 Motor Trend US magazine}. For now, let's focus only on cars with 4 cylinders. The black line just above 25 MPG is called median, i.e. the middle of the dataset. It indicates the point such that there is an equal probability of a MPG value falling above or below it. The top/bottom edges of the box indicate the upper and lower quartile, respectively. These quartiles tell us that there is only 25$\%$ chance a car with 4 cylinders will use more than 30 or less than 23 MPG. \textit{Did you realize that already 50$\%$ of our data lies within the box?} The horizontal lines above/below the box represent the maximum/minimum MPG values excluding outliers. Here, outliers are defined as MPG which appear less often than 10$\%$ of the cars with 4 cylinders, i.e. above 34 MPG or below 21 MPG.

\begin{figure}[h]
	\begin{center}
			\includegraphics[scale=0.3]{Parts/ch2/milleagech2}
	\end{center}
	\caption{Box-plot of miles per gallon by car cylinders. This data was extracted from the 1974 Motor Trend US magazine.}
	\label{fig:boxplot}
\end{figure} 

Note that the thick horizontal black lines inside the boxes represent the respective \textit{medians} for 4, 6 or 8 cylinders. Note that this value is different than the mean value. Here is why: the ``mean'' represents the ``average'' we all know. Simply sum up all the numbers and then divide them by the total number of occurrences. The ``median'' represents the ``middle'' of a ordered series of numbers. This is important because in practice we always find distributions which are highly skewed. In those cases, the median is more representative than the mean.

The mean or the median alone do not tell us the whole story about our data. We still have to understand the data \textit{distribution} and how the individual points are spread around the mean. To help us with that we will introduce the concept of standard deviation. In a nutshell, it represents by how much the individual samples differ from the mean. We depict it in Fig. \ref{fig:std}.

\begin{figure}[h]
	\begin{center}
			\includegraphics[scale=0.13]{Parts/ch2/stdch2}
	\end{center}
	\caption{Normal distribution where each band accounts for $\pm$ 1 standard deviation}
	\label{fig:std}
\end{figure} 

\newpage

The standard deviation is a measure of the dispersion of your data. That means a small standard deviation indicates little dispersion around the average. In contrast, a high standard deviation indicates significant dispersion around the average. As shown in Fig. \ref{fig:std}, the width of one standard deviation ``covers'' roughly 68$\%$ of the data dispersion. That number increases to over 99$\%$ for three standard deviations.

\section{Interpreting a plot, tracing a linear fit, and understanding R$^2$}

Another important topic we want to talk is how to interpret plots. As we show in Chapter \ref{Commun} plots are a fundamental way to communicate your data-science results within your company. For now, we will focus on extracting some interesting information from Fig. \ref{fig:linR}. Later, we go in much greater detail on how to design plots and interpret such model results.

\begin{figure}[h]
	\begin{center}
			\includegraphics[scale=0.35]{Parts/ch2/linregch2}
	\end{center}
	\caption{Miles per gallon versus the weight of the cars. The blue line draws a linear relation between MPG and the weight of the cars. The red dotted-lines represent the difference (residuum) between the linear fit and each data point. This data was extracted from the 1974 Motor Trend US magazine.}
	\label{fig:linR}
\end{figure} 

We observe a negative correlation between MPG and weight of the car. As the weight increases, MPG decreases. Note that this follows intuitively, since we expect heavier cars to use more fuel and therefore have lower MPG. The blue line represents a linear fit between the variables. The criterion to obtain the linear fit is to minimize the distance between the fit and the data points. This difference is called residuum (shown in red in Fig. \ref{fig:linR}). In Chapter \ref{numerics} we will cover the design of such simple numerical model in much greater detail. 

\newpage

For now let's assume we know how to implement such model - but how do we measure its performance? A simple way to achieve that is by calculating the so-called \textit{determination coefficient}, known as (R$^2$). R$^2$ is a metric that uses the residua information to determine if a linear fit is acceptable or not. It ranges from 0 to 1, where zero means no correlation between variables and 1 means a perfect fit. In the case shown in Fig. \ref{fig:linR} we find that $R^2=0.75$, indicating a good linear fit between MPG and weight of the car. In other words, the weight of the car can explain 75$\%$ of the MPG's variance. Generally speaking we define $R^2>0.5$ as acceptable. 

\section{Histograms}

The last topic we want to cover in this Chapter is histograms. A histogram is nothing more than a bar plot whose area represents the frequency of a given variable. In Fig. \ref{fig:histch2} we can visualize a histogram showing the frequency of the MPG variable:

\begin{figure}[h]
	\begin{center}
			\includegraphics[scale=0.7]{Parts/ch2/histch2}
	\end{center}
	\caption{Histogram showing the MPG frequency distribution. The red line emphasize its distribution. This data was extracted from the 1974 Motor Trend US magazine.}
	\label{fig:histch2}
\end{figure} 

\newpage

Note that the width of the bars represents the class interval. In our case every bar shows the frequency for an interval equal to 5 MPG. We can conclude from this plot that most cars have a MPG usage between 15 and 20 followed by 20-25 MPG (see also the red line). This plot also triggers many more questions, such as why is the frequency of cars using 30-35 MPG higher if compared to 25-30 MPG? It is not  possible to answer this question by only looking at this plot. The correct strategy is to combine a number of different analyses and plot more variables to understand this behaviour. We will exemplify that in Chapters \ref{harvesting} and \ref{numerics}.
\cleardoublepage

\newpage
\chapter{Practicalities: Software Installations and AWS cloud}\label{P1}
Coen
\cleardoublepage

\newpage
\chapter{Practicalities: Programming language and libraries}\label{P2}
Coen
\cleardoublepage

\newpage
\chapter{Harvesting data}\label{harvesting}
In this Chapter you will be exposed to a few topics, which are fundamental to prepare your data for analysis. We start from the most important one: designing a good research question. The motto is: \textit{``think before start modeling''}. With a good question (or questions) in hand, we jump to collect, put together, and integrate data into our algorithms. We go through some of the tricks to describe and finally clean the data. In summary, after this chapter your data will be understood, clean, and ready for analysis.

\section{The research question}\label{RQ}

The foremost aspect determining the success of your project is your research question. There is a certain set of attributes characterizing a good research question:

\begin{itemize}
	\item Clarity
	\item Current importance
	\item Searchability
	\item Simplicity
\end{itemize}

From all these aspects, clarity is absolutely the most important. A good research question should have only one interpretation and it should be very clear (free of ambiguity). It sets the stage for the methodology development. The topic should also be important to you and to your company at the moment. It should bring benefit and motivate discussions so you can continue with further developments. It should support different points of view and generate various perspectives. Furthermore, your research question hast to be supported by data you own (or will very soon). You don't want to realize in the middle of your project that there's not enough data to conduct your research. And finally, the rule is the simpler the better. Break down the complexity of one big question in smaller achievable questions. With the research question(s) at hand we move to integrating different datasets into our algorithm.

\section{Data Integration}\label{dataM}

There are many ways data can be integrated into our algorithms. Here are a few examples:

\begin{itemize}
\item hard-coded (for testing purposes)
\item loaded from multiple external files (e.g. excel csv, text file)
\item read from a database (e.g. SQL)
\item imported from a cloud (e.g. Amazon S3 buckets)
\end{itemize}

The first one (hard-coded) is mostly used during the built of the algorithm. Think, for example, of testing and debugging. In that stage, we want to try a few sample artificial data points to understand exactly what our code is doing. In doing so, we can correct mistakes much easier (and faster) than if utilizing an entire data set. 

Real data very often comes in the form of the so-called .csv (comma-separated values) files. These files are exported from Excel and are a handy way to share small datasets. Nowadays, there are pre-build commands in any major programming language to read such files, e.g. read.csv() in \textbf{R} or the csv.reader() in \textbf{Python}.

Another very common way to acquire data is by directly reading it from a database (e.g. SQL/Mongo) into your algorithm. There are connectors (e.g. ODBC for SQL) or libraries (pymongo, RMongo), which allow seemly coupling with our numerical algorithms. The main advantage of using a ``query-language'' to import data into our algorithm is flexibility, i.e. we can alter/expand these queries very easily.

Lastly, nowadays much of our data lies on the cloud. Therefore, we also cover how to read data from the Amazon cloud - from an S3 bucket. A bucket is a container used to store unlimited amounts of data. Therefore, you can have one big bucket for all of your information or, separate buckets for different types of data. Again, integration with these buckets is simple for both \textbf{Python} and \textbf{R}. Next, we will do some data manipulation.

\section{Data Manipulation}\label{dataM} 

The first thing we want to do after loading our data into the system is to be able to look at it. That means quickly describing it and later cleaning it. We quickly glance at both processes in this Section.

\subsection{Describing your data}\label{Analy.1}

Despite how big is your data it is fundamental for the ``human component'' in data science that we are able to visualize the data. That means inspect the first rows to have an impression of how the data looks. Check the column names to ensure they are properly assigned. Identify columns which data is not correct or simply useless, e.g. only zeros/blanks. These will give us already some insight on how to first handle the data. 

Something we should always do is to take advantage of some built-in functions in most programming languages. For this first inspection functions like \textit{describe} \textbf{(python)} and \textit{summary} \textbf{(R)} will do exactly what we want. They will calculate summary statistics of each column (even if column data types are variable, or some columns have no information). 

Another very important step is to plot your data. This is a very good way to actually see how your data looks like. Plot the columns against each other to look for correlations. Plot it as a time line to see changes or temporal patterns. Plot multiple columns together to investigate (dis)similarities during a certain period. Anyway, plot it and plot it more!

\subsection{Cleaning your data}\label{Analy.2}

More than often we find that our raw datasets are full of empty cells and ``NULL'' values, nonsense characters or ``zeros''. These datasets may miss headers, contain wrong data types, encoding, etc. When dealing with messy datasets the most important rule is twofold:

\begin{itemize}
\item each variable should be saved in its own column
\item each observation is saved in its own row
\end{itemize}

These rules make a what is called tidy dataset. If your dataset is tidy, you can easily perform comparisons amongst columns, such as ``bigger/smaller than average'' or ``equals to''. 


http://www.rstudio.com/wp-content/uploads/2015/02/data-wrangling-cheatsheet.pdf

Apart from that, as a rule of the thumb, columns with more than 50$\%$ empty/blank/noise can be safely deleted \footnote{50$\%$ is not a hard criterion but can be use as a first ``guesstimate''.}. In addition, we can always subset our data, or bind columns/rows from another dataset. We will practice that a lot during the course.  
\cleardoublepage

\newpage
\chapter{Describing and Modeling Data}\label{numerics}
%%https://rpubs.com/davoodastaraky/mtRegression

%http://rstudio-pubs-static.s3.amazonaws.com/20516_29b941670a4b42688292b4bb892a660f.html

In this Chapter we finally get our hands dirty with some real data! After learning how to acquire and import data, here we put that all in practice in our first data analysis case. This chapter is very hands-on, so we will present a complete and self-contained data science problem\footnote{See the works of Ryan Quan and Davood Astaraky for further reference}. We break down our investigation into small pieces. First, we have a quick look at our data, to then explore it in more detail. Finally, we design a numerical model to estimate some features of our dataset. 

\section{Descriptive data analysis}\label{initDA}

The case we present here is a classic one. We want to know if cars with a manual transmission are more fuel efficient than automatic-transmission cars. Following our methodology we first define our research questions. In this case, they are:
 
\begin{itemize}
\item Is there a difference in miles per gallon (MPG) between automatic and manual transmission cars?
\item Can we quantify the difference?
\end{itemize}

The data was extracted from the 1974 Motor Trend US magazine. It is called (\textit{mtcars}) and it comprises fuel consumption and 10 other aspects of car design and performance for 32 cars (73–74 models). Let's have a good look at what variables we have available, first:

\begin{figure}[ht]
	\begin{center}
			\includegraphics[scale=0.6]{Parts/numerics/desc1}
	\end{center}
	\caption{List of variables present in the mtcars dataset.}
	\label{fig:desc1}
\end{figure}

Also the first six records of the dataset are shown below:

\begin{figure}[ht]
	\begin{center}
			\includegraphics[scale=0.5]{Parts/numerics/desc2}
	\end{center}
	\caption{Six complete records. Every row represents a unique car model. The attributes names are described in Fig. \ref{fig:desc1}}
	\label{fig:desc2}
\end{figure}

\section{Exploratory data analysis}\label{ExpDA}

After this quick summary it is time to explore the dataset. Let's begin by looking at the MPG frequency distribution. \newpage

\begin{figure}[ht]
	\begin{center}
			\includegraphics[scale=0.5]{Parts/numerics/hist1}
	\end{center}
	\caption{Frequency distribution of MPG}
	\label{fig:hist1}
\end{figure}

We observe that the MPG distribution falls more or less under the normal curve (blue). Moreover, there are no apparent outliers. The whole range is contained within 10-35 MPG. That's good news for us! We can now plot the MPG for automatic and manual cars separately. What plot is better than a box-plot in this case?

\begin{figure}[ht]
	\begin{center}
			\includegraphics[scale=0.35]{Parts/numerics/bplot1}
	\end{center}
	\caption{MPG by transmission type}
	\label{fig:bplot1}
\end{figure}

Indeed, as shown in Fig. \ref{fig:hist1} the range which the MPG is distributed lies within 10-35. However, the box-plot separates the manual from automatic cars, allowing us to see that there is a (big?) difference between the two classes. Apparently, automatic cars have a lower MPG, following common sense. The medians are 17 and 23 MPG for automatic and manual, respectively. In addition, 50$\%$ of the automatic cars have an MPG confined within the range 15-20 MPG. This range is somewhat wider for manual cars ($\approx$ 20-30 MPG). Now, we have to prove that the difference we observe is not due to random chance, i.e. we just picked a bunch of automatic cars that not efficient and compared to very efficient manual cars. To make sure that does not happen we can use a statistical test. Here, we will use the so-called ``two Sample t-test''. 

The results are shown below:

\begin{figure}[ht]
	\begin{center}
			\includegraphics[scale=0.5]{Parts/numerics/ttest}
	\end{center}
%	\caption{}
	\label{fig:ttest}
\end{figure}

This test proves that cars with an automatic transmission use more fuel than cars with a manual transmission. The p-value ($0.001$) shows the probability that this apparent difference could appear by chance. Based on the very p-value the probability is negligible. The confidence intervals (3.2 - 11.3 MPG) describe the range of how much higher is the MPG in automatic cars, if compared to in manual cars.

Now, what if we want to design a model to estimate MPG based on the other variables we have in our dataset?

%https://www.datacamp.com/community/tutorials
%https://www.datacamp.com/community/tutorials/importing-data-r-part-two

\section{Regresion analysis}\label{RegDA}

What is the simplest model we can design? We'd say: what about a linear model that, given the type of transmission (automatic or manual), returns the MPG? Something like:

\begin{equation}
y = ax + b,
\end{equation}

\noindent where ``x'' is the transmission, ``y'' is the consumption in MPG, and ``a'' and ``b'' are numerical coefficients. That sounds simple enough, right? Let's have a look at the results of such linear regression:

\begin{figure}[ht]
	\begin{center}
			\includegraphics[scale=0.5]{Parts/numerics/linR}
	\end{center}
%	\caption{}
	\label{fig:linR}
\end{figure}

We are somewhat familiar with these numbers already. 17.1 is just the mean for automatic cars and the slope (7.2) is the difference, in MPG, between manual and automatic cars. The p-value is also very small, indicating that there's a significant difference between automatic and manual cars. However, if you calculate the R$^2$ of this fit you will get a very low 0.36. These results, together with the information we already have from the box-plots, make this model a little redundant - besides explaining only 36$\%$ of the variance.

Our previous results are calling for a more robust model. We will try a multi-variate linear model. In order to determine which predictors one should use, we create a correlation matrix for the MPG variable. Do it yourself and you will see that \textit{wt}, \textit{cyl}, \textit{disp}, and \textit{hp} are highly correlated with \textit{MPG} ($\left|0.7\right|$ or higher).

Let's do a full stop here: does that make sense? \textit{wt} and \textit{hp} make very good sense to us; heavier cars and more powerful cars should have lower MPG values. Note, however, that \textit{disp} and \textit{cyl} are highly correlated with each other. Due to that we decided to exclude them from our analysis. Summarizing we are making a model that looks like the following:

\begin{equation}
y = ax_1 + bx_2 + cx_3 + d,
\end{equation}

where $x_1,x_2,x_3$ are $am,wt$, and $hp$ respectively, and $a,b,c$, and $d$ are numerical coefficients to be determined by the model.

Since this Chapter is hands-on we will design this model together. The results are surprising. By introducing \textit{wt} and \textit{hp} to our modelling framework we obtain a R$^2$ of 0.84. This is much higher than $0.36$, obtained with the simple linear regression. This model predicts that, on average, manual transmission cars perform on around 2 MPGs more than automatic transmission cars.

If we come back and answer our research questions. Yes, there's a significant difference between automatic and manual transmission cars. We are also able to quantify this difference: 2 MPG. This is just an example of how thinking and modelling-design walk hand-in-hand. 

In the next Chapter we will dive on some machine learning algorithms and techniques.




\cleardoublepage

\chapter{Machine learning tools}\label{MLearn}

\section{Supervised methods}
\section{unsupervised methods}

\section{Titanic exploration}


\chapter{Databases}\label{databases}
\section{SQL}
\section{NoSQL}
\subsection{MongoDB}
\chapter{Big Data}\label{bigData}
\section{Map Reduce}\label{mapR}
\subsection{Hadoop}

\chapter{Communicating Results}\label{Commun}
\section{Visuals}\label{Visuals}

%supervised learning (trees and forests, nearest neighbor, regression)
%optimization (gradient descent)
%unsupervised learning (clustering, )





%http://work.caltech.edu/telecourse


%\begin{equation}
%T=\frac{1}{2}\cdot \sum{(X_i - X_{i-1})\cdot(Y_i+Y_{i-1})}
%\label{eq:T}
%\end{equation}
%
%\
%   
%
%\begin{figure}[h]
%
%	\begin{center}
%		$\begin{array}{c@{\hspace{1in}}c}
%%			\includegraphics[scale=0.25]{Simpsons} &
%%			\includegraphics[scale=0.45]{Simpson} \\ [0.4cm]
%			\mbox{\bf (Homer Simpson)} & \mbox{\bf (Metodo de Simpson)}
%		\end{array}$
%	\end{center}
%	\caption{Metodo de Simpson}
%	\label{fig:Simpson}
%\end{figure}
%
%\pagebreak
%
%
%\begin{eqnarray}
%S=1474379,23\;pes^2  \nonumber \\
%S=33,85 \;acres
%\label{eq:}
%\end{eqnarray}

\end{document}
